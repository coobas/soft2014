%%
%% This is file `elsarticle-template-num.tex',
%% generated with the docstrip utility.
%%
%% The original source files were:
%%
%% elsarticle.dtx  (with options: `numtemplate')
%% 
%% Copyright 2007, 2008 Elsevier Ltd.
%% 
%% This file is part of the 'Elsarticle Bundle'.
%% -------------------------------------------
%% 
%% It may be distributed under the conditions of the LaTeX Project Public
%% License, either version 1.2 of this license or (at your option) any
%% later version.  The latest version of this license is in
%%    http://www.latex-project.org/lppl.txt
%% and version 1.2 or later is part of all distributions of LaTeX
%% version 1999/12/01 or later.
%% 
%% The list of all files belonging to the 'Elsarticle Bundle' is
%% given in the file `manifest.txt'.
%% 

%% Template article for Elsevier's document class `elsarticle'
%% with numbered style bibliographic references
%% SP 2008/03/01

%\documentclass[5p,12pt]{elsarticle}
\documentclass[preprint,5p,times]{elsarticle}

%% Use the option review to obtain double line spacing
%% \documentclass[authoryear,preprint,review,12pt]{elsarticle}

%% Use the options 1p,twocolumn; 3p; 3p,twocolumn; 5p; or 5p,twocolumn
%% for a journal layout:
%% \documentclass[final,1p,times]{elsarticle}
%% \documentclass[final,1p,times,twocolumn]{elsarticle}
%% \documentclass[final,3p,times]{elsarticle}
%% \documentclass[final,3p,times,twocolumn]{elsarticle}
%% \documentclass[final,5p,times]{elsarticle}
%% \documentclass[final,5p,times,twocolumn]{elsarticle}

%% if you use PostScript figures in your article
%% use the graphics package for simple commands
\usepackage{graphics}
%% or use the graphicx package for more complicated commands
%% \usepackage{graphicx}
%% or use the epsfig package if you prefer to use the old commands
%% \usepackage{epsfig}

%% The amssymb package provides various useful mathematical symbols
\usepackage{amssymb}
%% The amsthm package provides extended theorem environments
%% \usepackage{amsthm}

%% The lineno packages adds line numbers. Start line numbering with
%% \begin{linenumbers}, end it with \end{linenumbers}. Or switch it on
%% for the whole article with \linenumbers.
\usepackage{lineno}


\usepackage{multirow} %&&&&&&&&&&&&&&&&&&&&&&&&&&&&&&&&&&
\usepackage{hhline}% http://ctan.org/pkg/hhline

\usepackage[utf8]{inputenc}
\usepackage{fancyvrb}

\usepackage{hyperref}

\usepackage{booktabs}

%\journal{Nuclear Physics B}
\journal{Fusion Engineering and Design}

\begin{document}

\begin{frontmatter}

%% Title, authors and addresses

%% use the tnoteref command within \title for footnotes;
%% use the tnotetext command for theassociated footnote;
%% use the fnref command within \author or \address for footnotes;
%% use the fntext command for theassociated footnote;
%% use the corref command within \author for corresponding author footnotes;
%% use the cortext command for theassociated footnote;
%% use the ead command for the email address,
%% and the form \ead[url] for the home page:
%% \title{Title\tnoteref{label1}}
%% \tnotetext[label1]{}
%% \author{Name\corref{cor1}\fnref{label2}}
%% \ead{email address}
%% \ead[url]{home page}
%% \fntext[label2]{}
%% \cortext[cor1]{}
%% \address{Address\fnref{label3}}
%% \fntext[label3]{}

\title{Validation of equilibrium tools on the COMPASS tokamak}

%% use optional labels to link authors explicitly to addresses:
 \author[label1]{J.~Urban}
 \author[label2]{L.C.~Appel}
 \author[label3]{J.F.~Artaud} 
 \author[label4]{B.~Faugeras} 
 \author[label1,label5]{J.~Havlicek}
 \author[label1]{M.~Komm}
 \author[label2]{I.~Lupelli} 
 \author[label1,label5]{M.~Peterka} 
 % \address[label1]{Institute of Plasma Physics AS CR, v.v.i., Association EURATOM / IPP.CR, Za~Slovankou 3, 182 00 Praha 8, Czech Republic}
 \address[label1]{Institute of Plasma Physics AS CR, v.v.i., Za~Slovankou 3, 182 00 Praha 8, Czech Republic}
 \address[label2]{CCFE, Culham Science Centre, Abingdon, Oxfordshire, UK}
 \address[label3]{CEA, IRFM, F-13108 Saint Paul Lez Durance, France}
 \address[label4]{Laboratoire J.A. Dieudonné, UMR 7351, Université de Nice Sophia-Antipolis, Parc Valrose, 06108
Nice Cedex 02, France}
 \address[label5]{Department of Surface and Plasma Science, Faculty of Mathematics and Physics, Charles University in Prague, V Hole\v{s}ovi\v{c}k\'ach 2, 180~00 Praha 8, Czech Republic}

\section{Abstract}

Various MHD (magnetohydrodynamic) equilibrium tools, some of which being recently developped or considerably updated, are used on the medium-size COMPASS tokamak \cite{P_nek_2006}. MHD equilibrium is a fundamental property of the tokamak plasma, whose knowledge is required for many diagnostics and modelling tools. Proper benchmarking and validation of equilibrium tools is thus key for interpretting and plannig tokamak experiments. We present here benchmarks and comparisons to experimental data of the EFIT++ reconstruction code [?], the free-boundary equilibrium code FREEBIE [?], and the rapid plasma boundary reconstruction part of the EQUINOX code \cite{doi:10.1016/j.jcp.2011.04.005}. We demonstrate that FREEBIE can calculate the equilibrium and corresponding poloidal field (PF) coils currents for given plasma parameters. Both EFIT++ and EQUINOX can reconstruct equilibria generated by FREEBIE from synthetic diagnostic data (including an artificial noise). Optimum reconstruction parameters are estimated; in addition, possible enhancements using more diagnostics are discussed and simulated using synthetic diagnostics. FREEBIE can also calculate the temporal evolution of the poloidal field coils currents for a whole plasma scenario.

The accuracy of the reconstruction depends on the noise level and certain numerical parameters. For example, EFIT++ reconstruction with a low-order $p'\left(\psi\right)$ and $ff'\left(\psi\right)$ is not perfect even with no noise. The recent EFIT++ version 6 is several times faster compared to version 4.
% \begin{abstract} %% Text of abstract 
% Various MHD (magnetohydrodynamic) equilibrium tools, some of which being recently developed or considerably updated, are used on the medium-size COMPASS tokamak [R. Pánek et al., Czech J Phys 56, B125, 2006]. MHD equilibrium is a fundamental property of the tokamak plasma, whose knowledge is required for many diagnostics and modelling tools. Proper benchmarking and validation of equilibrium tools is thus key for interpreting and planning tokamak experiments. We present here benchmarks and comparisons to experimental data of the EFIT++ reconstruction code [L.C. Appel et al., to be submitted to Nucl. Fusion], the free-boundary equilibrium code FREEBIE  [J.-F. Artaud, S.H. Kim, EPS 2012, P4.023], and a rapid plasma boundary reconstruction code VacTH [B. Faugeras et al., PPCF 2014, accepted]. We demonstrate that FREEBIE can calculate the equilibrium and corresponding poloidal field (PF) coils currents for given plasma parameters. Both EFIT++ and VacTH can reconstruct equilibria generated by FREEBIE from synthetic diagnostic data (including an artificial noise) and hence might be suitable for real-time control. Optimum reconstruction parameters are estimated; in addition, possible enhancements using more diagnostics are discussed and simulated using synthetic diagnostics. FREEBIE can also calculate the temporal evolution of the poloidal field coils currents for a whole plasma scenario.
% \end{abstract}

 
\begin{keyword}
%% keywords here, in the form: keyword \sep keyword
tokamak  \sep equilibrium \sep COMPASS 
%% PACS codes here, in the form: \PACS code \sep code
\PACS 52.55.Fa \sep 07.05.Kf \sep 07.05.Hd 
% Tokamaks, spherical tokamaks; Data management; Data acquisition
%\sep ???
%% MSC codes here, in the form: \MSC code \sep code
%% or \MSC[2008] code \sep code (2000 is the default)
\end{keyword}

\end{frontmatter}


%!TEX root = JUrban_SOFT2014.tex

\section{Introduction} % (fold)
\label{sec:introduction}

We report here on validation and verification of tokamak equilibrium tools used for the COMPASS tokamak \cite{compass2006}. We particularly focus on fundamental global plasma parameters and the shapes of magnetic flux surfaces, which are crucial in diagnostics interpretation and other analyses. 
EFIT++ \cite{efitpp2006} is used for routine equilibrium reconstruction on COMPASS. FREEBIE \cite{freebie2012} is a recent free-boundary equilibrium code; FREEBIE enables predictive equilibrium calculation consistent with the poloidal field (PF) components of the tokamak. In this study, FREEBIE is used in the so-called inverse mode, which predicts PF coils currents from a given plasma boundary and profiles. The third code employed in this study is VacTH \cite{vacthref}, which provides a fast reconstruction of the plasma boundary from magnetic measurements using a toroidal harmonics basis.

In order to verify and validate the aforementioned tools, we analyse EFIT++ and VacTH reconstructions of equilibria constructed with FREEBIE. Synthetic diagnostics (e.g., magnetic probes or flux loops) with optional artificial errors provide inputs for the reconstructions. 

% section introduction (end)

%!TEX root = JUrban_SOFT2014.tex

\section{Verification and validation procedure} % (fold)
\label{sec:procedure}

Reliable MHD equilibrium reconstruction is very important for tokamak exploitation. Numerous diagnostics and subsequent analyses require as inputs equilibrium properties such as flux surface geometry, magnetic field, stored energy, internal inductance etc. We have set up a set of benchmarking tasks, which verify and validate equilibrium tools that are currently employed on COMPASS. The procedure is fundamentally following:

\begin{enumerate}
	\item Equilibrium reconstruction of selected experimental cases using EFIT++.
	\item Recalculate the equilibria using FREEBIE in inverse mode.
	\item Optionally alter the equilibria in FREEBIE using e.g. experimental pressure profiles.
	\item Reconstruct FREEBIE equilibria using EFIT++ and VacTH with various parameters and artificial input noise.
\end{enumerate}

The first step employs a routine EFIT++ set-up for COMPASS with heuristically tuned parameters. In addition to the total plasma current $I_\mathrm{p}$ and the currents in individual PF circuits, 16 partial Rogowski coils and 4 flux loops are employed in this reconstruction and $p'$ and $FF'$ are assumed to be linear functions of the poloidal flux $\psi$.
FREEBIE inputs $I_\mathrm{p}$, $p'\left( {\bar \psi } \right)$ and $FF'\left( {\bar \psi } \right)$ profiles, the plasma boundary coordinates and an initial guess for the PF coils currents. Here, $p$ is the plasma pressure, $F = RB_\phi$ and $\bar\psi$ is the normalized poloidal magnetic flux ($\bar\psi = 0$ on the magnetic axis and $\bar\psi = 1$ on the plasma boundary). FREEBIE then seeks a solution to the Grad-Shafranov equation, including the PF coils currents, which minimizes the given plasma shape constraint. (This regime is called inverse mode.) FREEBIE can naturally output arbitrary synthetic diagnostics. We use here additional 24 poloidally and 24 radially oriented partial Rogowski coils (which are actually mounted on COMPASS) and an artificial set of 16 flux loops located at the same positions as the basic magnetic probes.

% section procedure (end)

%!TEX root = JUrban_SOFT2014.tex

\section{Results} % (fold)
\label{sec:results}

We have selected five time slices from COMPAS shots 4275 and 6962 (i.e. 10 cases in total) for the analysis. These cases include circular, elongated and diverted plasmas with different currents. A comparison of plasma shapes for shot 4275 is shown in Fig. \ref{fig:ex4275}. Numerical values of reconstruction errors are presented in Table \ref{table:ex4275}. We can observe a very good agreement between the original equilibrium and the reconstructed shapes. In this case, FREEBIE was using linear $p'$ and $FF'$ polynomials so that the EFIT++ model agrees with the target data. VacTH uses 8 magnetic probes and 16 flux loops. As we discuss later, flux loops are essential for reliable VacTH results. Even global kinetic properties are well reconstructed in EFIT++; the largest error around 10~\% is in $l_{\mathrm i}$ (i.e. basically in the toroidal current density profile).

\begin{table*}
\centering

\begin{tabular}{lrrrrrrrrrrrrr}
\toprule
   code &  time &  $\Delta R_{\mathrm in}$ &  $\Delta R_{\mathrm out}$ &  $\Delta Z_{\mathrm min}$ &  $\Delta Z_{\mathrm max}$ &  $\delta \kappa$ &  $E_\mathrm{mp}$ &  $E_\mathrm{fl}$ &  $\delta W$ &  $\delta l_{\mathrm i}$ &  $\delta \beta_{\mathrm p}$ &  $\delta q_0$ &  $\delta q_{95}$ \\
\midrule
 EFIT++ &  0.97 &              0 &            0.002 &        0.001 &        0.001 &                  0.003 &       0.001 &      0.0009 &          0.04 &           0.09 &              0.03 &           0.03 &           0.007 \\
 EFIT++ &  0.99 &          4e-05 &            9e-05 &        0.001 &        0.001 &                  0.004 &       0.001 &      0.0006 &          0.04 &           0.09 &              0.04 &           0.02 &           0.005 \\
 EFIT++ &  1.02 &         0.0005 &           0.0002 &        0.002 &        0.002 &                  0.008 &       0.002 &       0.002 &          0.02 &            0.1 &              0.02 &           0.02 &           0.009 \\
 EFIT++ &  1.05 &          0.001 &           0.0008 &        4e-05 &       0.0003 &                  0.004 &       0.003 &       0.005 &          0.01 &           0.07 &              0.02 &           0.01 &           0.009 \\
 EFIT++ &   1.1 &          0.001 &           0.0005 &        0.005 &       0.0002 &                  0.005 &       0.004 &       0.002 &          0.04 &           0.09 &              0.03 &           0.02 &            0.05 \\
  VacTH &  0.97 &              0 &            0.001 &       0.0006 &       0.0002 &                  0.002 &       8e-07 &       7e-05 &           nan &            nan &               nan &            nan &             nan \\
  VacTH &  0.99 &          4e-05 &           0.0004 &        0.001 &       0.0008 &                  0.004 &       4e-07 &      0.0001 &           nan &            nan &               nan &            nan &             nan \\
  VacTH &  1.02 &          0.002 &           0.0008 &        0.005 &        0.002 &                   0.01 &       2e-06 &       0.002 &           nan &            nan &               nan &            nan &             nan \\
  VacTH &  1.05 &          0.006 &            0.002 &        5e-05 &        0.002 &                   0.01 &       3e-06 &        0.02 &           nan &            nan &               nan &            nan &             nan \\
  VacTH &   1.1 &          0.005 &            0.002 &        0.002 &        0.003 &                   0.02 &       2e-06 &       0.003 &           nan &            nan &               nan &            nan &             nan \\
\bottomrule
\end{tabular}

\caption{Errors for the same cases as in Fig. \ref{fig:ex4275}.}
\label{table:ex4275}
\end{table*}

\begin{figure*}
\centering   %\begin{center}
\hfill{}
\includegraphics[width=18cm]{figures/example_4275.pdf}
\hfill{}
%\end{center}
\caption{Contours of $\bar\psi=\left(0.25,0.5,0.75,1\right)$, reconstruction from FREEBIE data, shot 4275. EFIT++ parameters: $n_\mathrm{mp} = 16$, $n_\mathrm{fl} = 4$, $n_{p'} = n_{FF'} = 1$. VacTH parameters: $n_\mathrm{mp} = 8$, $n_\mathrm{fl} = 16$, $n_{\mathrm p} = n_{\mathrm q} = 5$.}
\label{fig:ex4275}
\end{figure*}


The second shot for the comparison is 6962, which has been chosen because Thomson scattering (TS) profiles are available. On top of the same exercise as for 4275, we have calculated with FREEBIE equilibria with experimental TS pressure profiles. These equilibria of course no longer feature linear $p'$ and $FF'$ profiles. Results with smoothed TS profiles and an artificial noise in magnetic probes and flux loops with $\epsilon = 0.03$ are shown in Fig. \ref{fig:ex6962}.

\begin{figure*}
\centering   %\begin{center}
\hfill{}
\includegraphics[width=18cm]{figures/example_6962_TS_noise.pdf}
\hfill{}
%\end{center}
\caption{Contours of $\bar\psi=\left(0.25,0.5,0.75,1\right)$, reconstruction from FREEBIE data, shot 4275. EFIT++ parameters: $n_\mathrm{mp} = 16$, $n_\mathrm{fl} = 4$, $n_{p'} = n_{FF'} = 1$. VacTH parameters: $n_\mathrm{mp} = 8$, $n_\mathrm{fl} = 16$, $n_{\mathrm p} = n_{\mathrm q} = 5$.}
\label{fig:ex6962}
\end{figure*}


% section results (end)

%!TEX root = JUrban_SOFT2014.tex

\section{Conclusions} % (fold)
\label{sec:conclusions}

Two new codes---FREEBIE and VacTH---have been successfully set up on COMPASS, which enabled to perform an extensive cross-benchmarking and validation of free-boundary equilibrium tools. We show that FREEBIE can predict equilibria that are consistent with EFIT++ reconstructions from experimental data. FREEBIE model equilibria, either with linear $p'$ and $FF'$ profiles or with pressure profiles from Thomson scattering diagnostic, then served to assess the credibility of EFIT++ reconstructions. 
% This has not been up to now possible.

We show that magnetic reconstruction EFIT++ with linear $p'$ and $FF'$ features a relatively good accuracy of 1 -- 2 cm in the plasma shape reconstruction but introduces a systematic error in internal plasma parameters, such as $W$, $l_{\mathrm i}$, $\beta_{\mathrm p}$ or $q_0$. The reconstruction properties can be significantly improved by using quadratic $p'$ for diverted plasmas, which removes the systematic error and also improves the LCFS reconstruction. EFIT++ converges in 100~\% cases in this regime.

Optimum parameters for VacTH have been estimated. In particular, the optimum number of harmonics is 4 otherwise VacTH fails to converge in many cases, even without any input error. 16 flux loops and only 8 magnetic must be used as VacTH input. With less flux loops or more magnetic probes the code performs significantly worse. We conclude that VacTH is a promising tool pertinent for a real-time feedback control of the plasma shape.

% section conclusions (end)


\section*{Acknowledgement}
This work has been carried out within the framework of the Contract of
Association between EURATOM and the IPP.CR.  The views and opinions do not necessarily reflect those
of the European Commission.
The work of J. Urban was supported by Czech Science Foundation grant 13-38121P,
the work at IPP AS CR by MSMT LM2011021.

% \appendix
% \input{Appendix.tex}

% \section*{References} % (fold)

\bibliographystyle{model1-num-names}
\bibliography{bibliography/biblio}{}

% \bibliographystyle{unsrt}

% \bibliographystyle{abbrvnat}

% \usepackage{natbib}
% \bibliographystyle{chicago}

\end{document}
\endinput
