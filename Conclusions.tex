%!TEX root = JUrban_SOFT2014.tex

\section{Conclusions} % (fold)
\label{sec:conclusions}

Two new codes---FREEBIE and VacTH---have been successfully set up on COMPASS, which enabled to perform an extensive cross-benchmarking and validation of free-boundary equilibrium tools. We show that FREEBIE can predict equilibria that are consistent with EFIT++ reconstructions from experimental data. FREEBIE model equilibria, either with linear $p'$ and $FF'$ profiles or with pressure profiles from Thomson scattering diagnostic, then served to assess the credibility of EFIT++ reconstructions. 
% This has not been up to now possible.

We show that magnetic reconstruction EFIT++ with linear $p'$ and $FF'$ features a relatively good accuracy of 1 -- 2 cm in the plasma shape reconstruction but introduces systematic errors both in the shape and in internal plasma parameters, such as $W$, $l_{\mathrm i}$, $\beta_{\mathrm p}$ or $q_0$. The reconstruction properties can be significantly improved by using quadratic $p'$ for elongated and divertor plasmas, which removes the systematic error and also improves the LCFS reconstruction. EFIT++ converges in 100~\% cases in this regime.

Optimum parameters for VacTH have been estimated. In particular, the optimum number of harmonics is 4 otherwise VacTH fails to converge in many cases, even without any input error. 16 flux loops and only 8 magnetic must be used as VacTH input. With less flux loops or more magnetic probes the code performs significantly worse. We conclude that VacTH is a promising tool pertinent for a real-time feedback control of the plasma shape.

% section conclusions (end)
